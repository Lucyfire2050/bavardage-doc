\documentclass[a4paper,11pt,french]{article}
\usepackage[utf8]{inputenc}

\usepackage[T1]{fontenc}
\usepackage[francais]{babel}
\usepackage[top=2cm, bottom=2cm, left=2cm, right=2cm, includeheadfoot]{geometry} %pour les marges
\usepackage{lmodern}
\usepackage{pict2e}
\usepackage{tikz}	
\usepackage{tikz-uml}
\usepackage{fancyhdr} % Required for custom headers
\usepackage{lastpage} % Required to determine the last page for the footer
\usepackage{extramarks} % Required for headers and footers
\usepackage{graphicx} % Required to insert images
\usepackage{tabularx, longtable}
\usepackage{color, colortbl}

\usepgflibrary{arrows} % for pgf-umlsd

\linespread{1.1} % Line spacing

% Set up the header and footer
\pagestyle{fancy}
\lhead{\textbf{\hmwkClass -- \hmwkSubject \\ \hmwkTitle \\ \hmwkDocName}} % Top left header
\rhead{\includegraphics[width=10em]{logo_univ.png}}
\lfoot{\lastxmark} % Bottom left footer
\cfoot{} % Bottom center footer
\rfoot{Page\ \thepage\ / \pageref{LastPage}} % Bottom right footer
\renewcommand\headrulewidth{0.4pt} % Size of the header rule
\renewcommand\footrulewidth{0.4pt} % Size of the footer rule

\setlength{\headheight}{40pt}

\newcommand{\hmwkTitle}{Chat sécurisé} % Assignment title
\newcommand{\hmwkClass}{Master 1 SSI } % Course/class
\newcommand{\hmwkAuthorName}{Jean-Baptiste Souchal} % Your name
\newcommand{\hmwkSubject}{Conduite de projet} % Subject
\newcommand{\hmwkDocName}{Spécification technique des besoins} % Document name

\newcommand{\version}{0.7} % Document version
\newcommand{\docDate}{03 janvier 2013} % Document date
\newcommand{\checked}{Julien Legras} % Checker name
\newcommand{\approved}{} % Approver name

\newcommand{\fiche}[9] {
	\noindent
\begin{tabular}{|p{3.5cm}| p{1cm} | p{3cm} | p{.5cm} | p{7cm}|} 
\hline
\rowcolor{blue}
\multicolumn{2}{|l|}{\color{white}\bfseries{Nom}} & \multicolumn{3}{l|}{\color{white}\bfseries{#1}}\\
\hline
\multicolumn{2}{|l|}{\bfseries{Acteurs concernés}} & \multicolumn{3}{m{10.5cm}|}{#2}\\
\hline
\multicolumn{2}{|l|}{\bfseries{Description}} & \multicolumn{3}{m{10.5cm}|}{#3}\\
\hline
\multicolumn{2}{|l|}{\bfseries{Préconditions}} & \multicolumn{3}{m{10.5cm}|}{#4}\\
\hline
\multicolumn{2}{|l|}{\bfseries{Evénements déclenchants}} & \multicolumn{3}{m{10.5cm}|}{#5}\\
\hline
\multicolumn{2}{|l|}{\bfseries{Conditions d'arrêt}} & \multicolumn{3}{m{10.5cm}|}{#6}\\
\hline
\rowcolor{gray}
\multicolumn{5}{|c|}{\bfseries{Description du flot d'événements principal}}\\
\hline
\rowcolor{gray}
\multicolumn{3}{|c|}{\bfseries{Acteur(s)}} & \multicolumn{2}{c|}{\bfseries{Système}}\\
\hline
\multicolumn{3}{|p{7.5cm}|}{#7} & \multicolumn{2}{p{7.5cm}|}{#8}\\
\hline
\multicolumn{2}{|l}{\bfseries{Flots d'exceptions}} & \multicolumn{3}{|p{11.5cm}|}{#9}\\
\hline
\end{tabular}
\\
}

\definecolor{gris}{rgb}{0.95, 0.95, 0.95}

\title{
\vspace{2in}
\textmd{\textbf{\hmwkClass :\ \hmwkTitle}}\\
\normalsize\vspace{0.1in}\small{Due\ on\ \hmwkDueDate}\\
\vspace{0.1in}\large{\textit{\hmwkClassInstructor\ \hmwkClassTime}}
\vspace{3in}
}

\author{\hmwkAuthorName}
\date{} % Insert date here if you want it to appear below your name


\begin{document}
\pagestyle{fancy}

\vspace*{5cm}
\begin{center}\textbf{\Huge{\hmwkDocName}}\end{center}
\vspace*{7cm}
	

\fcolorbox{black}{gris}{
\begin{minipage}{15cm}
\begin{tabularx}{10cm}{lXl}
	\bfseries{Version} & & \version\\
	& & \\
	\bfseries{Date} & & \docDate\\
	& & \\
	\bfseries{Rédigé par} & & \hmwkAuthorName \\
	& & \\
	\bfseries{Relu par} & & \checked \\
	& & \\
	\bfseries{Approuvé par} & & \approved \\
	& & \\
\end{tabularx}
\end{minipage}
}

\newpage

%Tableau de mises à jour
\vspace*{1cm}
\begin{center}
\textbf{\huge{MISES À JOUR}}\\
\vspace*{3cm}
	\begin{tabularx}{16cm}{|c|c|X|}
	\hline
	\bfseries{Version} & \bfseries{Date} & \bfseries{Modifications réalisées}\\
	\hline
	0.1 & 09/11/2011 & Création\\
	\hline
	0.2 & 23/11/12 & Rédaction des cas d'utilisation\\
	\hline
	0.3 & 28/11/12 & Modifications après retour client\\
	\hline
	0.4 & 09/12/12 & Ajout UC 17--18--19 et des règles de gestion\\
	\hline
	0.5 & 14/12/12 & Modifications après réunion du 12/12/12\\
	\hline
	0.6 & 29/12/12 & Modifications après réunion du 19/12/12\\
	\hline
	0.7 & 03/01/13 & Modifications après retour client\\
	\hline
	\end{tabularx}
\end{center}

%La table des matières
\clearpage
\tableofcontents
\clearpage

% OBJET
\section{Objet}
Développement d'un système permettant à plusieurs utilisateurs de s'authentifier et de communiquer de manière sécurisée :
\begin{itemize}
\item gestion de la création et de la suppression d'un compte utilisateur ;
\item création par un utilisateur d'une salle de discussion privée ;
\item ajout et suppression d'un utilisateur autorisé dans une salle privée
\item confidentialité, intégrité et authentification sur les messages échangés ;
\item non répudiation des messages ;
\item création d'une autorité de certification ;
\item demande de certificat pour l'accès à un salon privé et la communication sécurisée.
\end{itemize}

% DOCUMENTS APPLICABLES ET DE RÉFÉRENCE
\section{Documents applicables et de référence}
\begin{itemize}
\item IRC (RFC 2810 à 2813 d'avril 2000)
\end{itemize}

% TERMINOLOGIE
\section{Terminologie et sigles utilisés}
\begin{itemize}
\item \textbf{IRC (Internet Relay Chat)} ;
\item \textbf{RFC (Request For Comments)} ;
\item \textbf{Autorité de Certification (AC)} : gère la délivrance des certificats et des clés publique/privée (EO.4) ;
\item \textbf{Autorité d'enregistrement (AR)} : Organisme qui génère les certificats et effectue les vérifications d'usage sur les utilisateurs;
\item \textbf{Serveur de chat (Server)} : serveur accueillant des utilisateurs ;
\item \textbf{Serveur de chat sécurisé (Server S.)} : serveur accueillant des utilisateurs sécurisés (EO.5, EO.6, EO.7);
\item \textbf{Client de chat (Client)} : le client de chat permet l'envoi et l'affichage des messages entre utilisateurs (EO.2).
\item \textbf{Client sécurisé (Client S)} :  a pour mission de chiffrer et déchiffrer les messages (EO.1, EO.2, EO.3, EO.5, EO.6).
\item \textbf{Salon privé (Salon P.)} : salle de chat privée à l'intérieur du serveur de chat permettant à des utilisateurs sécurisés de communiquer ;
\item \textbf{Utilisateur (User)} : personne étant connectée sur un serveur de chat ;
\item \textbf{Utilisateur sécurisé (User S.)} : utilisateur ayant un certificat délivré par AC et des clés privée/publique permettant la création de salons privés et l'échange de messages sécurisés ;
\item \textbf{Administrateur (Admin)} : utilisateur sécurisé ayant créé un salon privé et disposant des droits d'administration sur ce salon privé ;
\item \textbf{Message privé (MP)} : un MP est un message (sécurisé ou non) à destination d'un unique utilisateur (sécurisé ou non).
\end{itemize}

% EXIGENCES FONCTIONNELLES
\section{Exigences fonctionnelles}
\subsection{Présentation de la mission du produit logiciel}
\begin{itemize}
\item [EF.1] : création d'un compte sécurisé ;
\item [EF.2] : suppression compte sécurisé par l'utilisateur ;
\item [EF.3] : suppression compte sécurisé par l'autorité de certification ;
\item [EF.4] : ajout certificat à un utilisateur par l'autorité de certification ;
\item [EF.5] : envoi de message sur salon général ;
\item [EF.6] : envoi de message sur salon privé ;
\item [EF.7] : rejoindre un serveur de chat sans authentification ;
\item [EF.8] : rejoindre un serveur de chat sécurisé avec authentification ;
\item [EF.9] : quitter un serveur de chat;
\item [EF.10] : quitter un serveur de chat sécurisé ;
\item [EF.11] : quitter un salon privé ;
\item [EF.12] : renouvellement des clefs des salons privés;
\item [EF.13] : déconnexion technique ;
\item [EF.14] : créer un salon privé ;
\item [EF.15] : rejoindre un salon privé;
\item [EF.16] : fermeture d'un salon privé;
\item [EF.17] : exclure utilisateur sécurisé d'un salon privé;
\item [EF.18] : invitation d'un utilisateur sécurisé dans un salon privé;
\item [EF.19] : envoi d'un MP non sécurisé;
\item [EF.20] : envoi d'un MP sécurisé;
\item [EF.21] : demande pour rejoindre un salon privé;
\item [EF.22] : un utilisateur génère ses propres clefs;
\item [EF.23] : mise en place d'une autorité de certification;
\item [EF.24] : authentification entre les différents acteurs du système entre chaque échange;
\item [EF.25] : certification du Server S. par AC;
\item [EF.26] : établir une clef de session entre un utilisateur et un serveur pour l'authentification mutuelle lors des échanges.
\end{itemize}

\begin{center}
\begin{tabular}{|l|p{5cm}|l|l|}
\hline
\bfseries{Id} & \bfseries{Intitulé} & \bfseries{Acteur(s)}&\bfseries{Priorité}\\
\hline
UC.1 & Création d'un compte utilisateur sécurisé & User / AC / AR / Server S&indispensable\\
\hline
UC.2 & Suppression d'un compte sécurisé par utilisateur sécurisé & User S / Server S / AC&indispensable\\
\hline
UC.3 & Révocation d'un certificat par AC & AC&indispensable\\
\hline
UC.4 & Envoi de message sur salon général & User / User S / Client / Server&indispensable\\
\hline
UC.5 & Envoi de message sur salon privé & User S / Admin / Client S / Server S&indispensable\\
\hline
UC.6 & Rejoindre un serveur de chat sécurisé avec authentification & User S / Server S / Client S&indispensable\\
\hline
UC.7 & Rejoindre un serveur de chat sans authentification & User / Server / Client&indispensable\\
\hline
UC.8 & Quitter un serveur de chat & User / Server / Client&indispensable\\
\hline
UC.9 & Quitter un serveur de chat sécurisé & User S / Server S / Client S&indispensable\\
\hline
UC.10 & Quitter un salon privé & User S / Server S / Client S&indispensable\\
\hline
UC.11 & Déconnexion serveurs & User / User S / Server / Server S&indispensable\\
\hline
UC.12 & Créer un salon privé & User S / Server S / Client S&indispensable\\
\hline
UC.13 & Rejoindre un salon privé & User S / Server S / Admin / Client S&indispensable\\
\hline
UC.14 & Fermeture d'un salon privé & Admin / Server S / Client S&indispensable\\
\hline
UC.15 & Exclure utilisateur sécurisé d'un salon privé & Admin / Server S / Client S&important\\
\hline
UC.16 & Invitation d'un utilisateur sécurisé dans un salon privé & Admin / Server / User S / Client S&indispensable\\
\hline 
UC.17 & Envoi d'un message privé non sécurisé & User / User S / Client / Server&important\\
\hline
UC.18 & Envoi d'un message privé sécurisé & User S / Client S / Server&indispensable\\
\hline
UC.19 & Demande d'ajout dans un salon privé & User S / Admin / Server S / Client S&important\\
\hline
\end{tabular}
\end{center}

\newpage
\begin{itemize}
\item [RG.1] Le nom d'utilisateur doit être disponible et valide (\verb![A-Za-z0_9]+!)
\item [RG.2] Le message clair doit respecter la taille maximale d'un message (512 caractères)
\item [RG.3] Le nom du User S doit exister
\item [RG.4] Le certificat du User S doit être valide
\item [RG.5] Le nom du salon doit être disponible et valide (\verb![A-Za-z0_9]+!)
\end{itemize}
\subsection{UC 1 : Création d'un compte Utilisateur sécurisé}

\fiche
	{Création d'un compte Utilisateur sécurisé} %nom
	{User / AC / AR / Serveur S} %acteurs concernés
	{Un utilisateur fait la demande de création d’un compte utilisateur sécurisé auprès de AC} %description
	{Avoir les prérequis pour obtenir le certificat, prérequis définis par AC} %préconditions
	{Envoi de la demande de création du compte sécurisé à AC} % événements déclenchants
	{} %conditions d'arret
	{\begin{itemize}  %flot d'événements (acteurs)
		\item [1.] l’utilisateur génère ces clés
		\item [2.] l’utilisateur remplit le formulaire de demande de création d’un compte utilisateur sécurisé 
		\item [3.] l’utilisateur envoie le formulaire à AC  
	 \end{itemize}
	} 
	{\begin{itemize}  %flot d'événements (systeme)
		\item []
		\item []
		\item [4.] AC traite la demande de création du compte sécurisé
		\item [5.] envoi par l'AC du certificat à l'utilisateur
	\end{itemize}
	 }
	{RG.1} %Flots d'exceptions

\subsection{UC 2 : Suppression d’un compte sécurisé par un Utilisateur}

\fiche
	{Suppression d’un compte sécurisé par un Utilisateur} %nom
	{User S. / Server S. / AC} %acteurs concernés
	{Un utilisateur sécurisé supprime son compte du serveur de chat sécurisé} %description
	{Avoir un compte utilisateur sécurisé sur le serveur de chat sécurisé} %préconditions
	{Suppression du compte utilisateur sécurisé dans le registre du serveur de chat sécurisé} % événements déclenchants
	{} %conditions d'arret
	{\begin{itemize}  %flot d'événements (acteurs)
		\item [1.] l’utilisateur sécurisé fait la demande de suppression de son compte
		\item []
		\item []
		\item []
		\item []
		\item []
		\item [5.] l’utilisateur est déconnecté du serveur de chat sécurisé et du client de chat
	 \end{itemize}
	} 
	{\begin{itemize}  %flot d'événements (systeme)
		\item []
		\item []
		\item [2.] le serveur de chat sécurisé supprime le compte utilisateur sécurisé de ses registres
		\item [3.] le serveur de chat sécurisé arrête la connexion en cours de l’utilisateur sécurisé
		\item [4.] suppression des certificats de l’utilisateur par AC
	 \end{itemize}
	 }
	{} %Flots d'exceptions

\subsection{UC 3 : Révocation d’un certificat par AC}

\fiche
	{Révocation d’un certificat par AC} %nom
	{AC} %acteurs concernés
	{Suppression d’un compte par AC pour non respect du règlement ou inactivité d’un utilisateur sécurisé} %description
	{Le compte utilisateur sécurisé ne respecte pas les règles de certification} %préconditions
	{Suppression du compte dans le registre du serveur de chat sécurisé} % événements déclenchants
	{} %conditions d'arret
	{\begin{itemize}  %flot d'événements (acteurs)
		\item [] 
	 \end{itemize}
	} 
	{\begin{itemize}  %flot d'événements (systeme)
		\item [1.] AC révoque le certificat de l'utilisateur concerné.
	 \end{itemize}
	 }
	{} %Flots d'exceptions

\subsection{UC 4 : Envoi de message sur salon général}

\fiche
	{Envoi de message sur salon général} %nom
	{User / User S. / Client / Server} %acteurs concernés
	{Un utilisateur (sécurisé) envoie un message sur le salon général du serveur de chat et le client affiche le message} %description
	{Être connecté sur le serveur de chat } %préconditions
	{Le client envoie sur le salon général du serveur de chat le message  et est lisible par tous les utilisateurs connectés sur ce serveur de chat} % événements déclenchants
	{} %conditions d'arret
	{\begin{itemize}  %flot d'événements (acteurs)
		\item [1.] Un utilisateur envoie un message sur le salon général du serveur de chat
	 \end{itemize}
	} 
	{\begin{itemize}  %flot d'événements (systeme)
		\item []
		\item [2.] le Client envoie le message au Server
		\item [3.] le serveur de chat reçoit et envoie le message au(x) Client(s) de chat des utilisateurs connectés sur le salon privé
		\item [4.] le message s’affiche dans le(s) Client(s) de chat de(s) utilisateur(s)
	 \end{itemize}
	 }
	{RG.2} %Flots d'exceptions

\subsection{UC 5 : Envoi de message sur salon privé}

\fiche
	{Envoi de message sur salon privé} %nom
	{User S. / Admin / Server / Client S} %acteurs concernés
	{Un utilisateur sécurisé envoie un message sur le salon privé} %description
	{L’utilisateur sécurisé est connecté sur le serveur de chat sécurisé et est dans le salon privé et être connecté sur le serveur de chat} %préconditions
	{Un message est envoyé dans le salon privé et est lisible par tous les utilisateurs sécurisés connectés sur ce salon privé} % événements déclenchants
	{} %conditions d'arret
	{\begin{itemize}  %flot d'événements (acteurs)
		\item [1.] un User S. envoi un message sur le salon privé
	 \end{itemize}
	} 
	{\begin{itemize}  %flot d'événements (systeme)
		\item []
		\item [2.] le Client S. chiffre le message
		\item [3.] le Client de chat envoie le message au serveur de chat
		\item [4.] le serveur de chat envoie le message sur le chat général du salon privé, et est visible par les User S. sécurisé connecté sur le salon privé
	 \end{itemize}
	 }
	{RG.2} %Flots d'exceptions

\subsection{UC 6 : Rejoindre un serveur de chat sécurisé avec authentification}

\fiche
	{Rejoindre un serveur de chat sécurisé avec authentification} %nom
	{User / User S. / Client / Server} %acteurs concernés
	{Un utilisateur sécurisé se connecte sur le serveur de chat sécurisé} %description
	{Avoir un compte sécurisé  sur le serveur de chat sécurisé} %préconditions
	{L’utilisateur sécurisé est connecté sur le serveur de chat sécurisé} % événements déclenchants
	{} %conditions d'arret
	{\begin{itemize}  %flot d'événements (acteurs)
		\item [1.] l’utilisateur sécurisé entre son nom d’utilisateur et se connecte au serveur de chat sécurisé
		\item [] 
		\item [] 
		\item [] 
		\item [] 
		\item [] 
		\item [] 
		\item [] 
		\item [] 
		\item []  
		\item [8.] l’utilisateur sécurisé est connecté aux Server S. et Server
	 \end{itemize}
	} 
	{\begin{itemize}  %flot d'événements (systeme)
		\item []
		\item [2.] le Client envoie la demande aux Server et Server S.
		\item [3.] le Server S. reçoit la demande de connexion général
		\item [4.] le Server S. vérifie les certificats du compte utilisateur sécurisé et authentifie l’utilisateur sécurisé
		\item [5.] le Server S. accepte la connexion de l’utilisateur sécurisé
		\item [6.] création d’un compte utilisateur temporaire par le Server
		\item [7.] connexion au Server
	 \end{itemize}
	 }
	{RG.3, RG.4} %Flots d'exceptions

\subsection{UC 7 : Rejoindre un serveur de chat sans authentification}

\fiche
	{Rejoindre un serveur de chat sans authentification} %nom
	{User / Server / Client} %acteurs concernés
	{Un utilisateur se connecte sur le serveur de chat} %description
	{Saisir un nom d’utilisateur valide} %préconditions
	{L’utilisateur est connecté sur le serveur de chat} % événements déclenchants
	{} %conditions d'arret
	{\begin{itemize}  %flot d'événements (acteurs)
		\item [1.] l’utilisateur entre un nom d’utilisateur et ce connecte au serveur de chat
		\item[]  
		\item[]  
		\item[]  
		\item[]  
		\item[]  
		\item[]  
		\item [6.] l’utilisateur est connecté au serveur de chat
	 \end{itemize}
	} 
	{\begin{itemize}  %flot d'événements (systeme)
		\item []
		\item []
		\item [2.] le Client envoie la demande au Server
		\item [3.] le serveur de chat reçoit la demande de connexion
		\item [4.] le serveur de chat vérifie la validité du nom d’utilisateur
		\item [5.] le serveur de chat accepte la connexion de l’utilisateur 
	 \end{itemize}
	 }
	{RG.1} %Flots d'exceptions

\subsection{UC 8 : Quitter un serveur de chat}

\fiche
	{Quitter un serveur de chat} %nom
	{User / Server / Client} %acteurs concernés
	{Un utilisateur quitte le serveur de chat} %description
	{Être connecté sur le serveur de chat } %préconditions
	{L’utilisateur est déconnecté du serveur de chat} % événements déclenchants
	{} %conditions d'arret
	{\begin{itemize}  %flot d'événements (acteurs)
		\item [1.] l’utilisateur quitte le serveur de chat
		\item []  
		\item []
		\item [4.] l’utilisateur est déconnecté du Server
	 \end{itemize}
	} 
	{\begin{itemize}  %flot d'événements (systeme)
		\item []
		\item [2.] le Client envoie la demande au Server
		\item [3.] le serveur de chat déconnecte l’utilisateur et supprime de ses registres l’utilisateur
	 \end{itemize}
	 }
	{Les certificats de l’utilisateur sécurisé ne sont pas valides} %Flots d'exceptions

\subsection{UC 9 : Quitter un serveur de chat sécurisé}

\fiche
	{Quitter un serveur de chat sécurisé} %nom
	{User S. / Server S. / Client} %acteurs concernés
	{Un utilisateur sécurisé quitte le serveur de chat sécurisé} %description
	{Être connecté sur le serveur de chat sécurisé} %préconditions
	{L’utilisateur est déconnecté du serveur de chat sécurisé} % événements déclenchants
	{} %conditions d'arret
	{\begin{itemize}  %flot d'événements (acteurs)
		\item [1.] l’utilisateur sécurisé quitte le serveur de chat sécurisé
		\item[]  
		\item[]  
		\item[]  
		\item[]  
		\item [6.] l’utilisateur est déconnecté su Server S (et Server si UC.10)
	 \end{itemize}
	} 
	{\begin{itemize}  %flot d'événements (systeme)
		\item []
		\item [2.] le Client envoie la demande au Server S.
		\item [3.] le Server reçoit la demande du Client
		\item [4.] Si User S. connecté a un salon privé => UC.10
		\item [5.] le serveur de chat sécurisé déconnecte l’utilisateur sécurisé
	 \end{itemize}
	 }
	{} %Flots d'exceptions

\subsection{UC 10 : Quitter un salon privé}

\fiche
	{Quitter un salon privé} %nom
	{User S. / Server S. / Client S.} %acteurs concernés
	{Un utilisateur sécurisé quitte le salon privé} %description
	{Être dans le salon de discussion privée} %préconditions
	{L’utilisateur sort du salon de discussion privée et les clés du salon sont modifiées puis renvoyées à tous les autres utilisateurs présents sur le salon de discussion privée} % événements déclenchants
	{} %conditions d'arret
	{\begin{itemize}  %flot d'événements (acteurs)
		\item [1.] User S. quitte le salon privé
		\item[]  
	 \end{itemize}
	} 
	{\begin{itemize}  %flot d'événements (systeme)
		\item []
		\item [2.] le Client S. envoie la demande au Server S.
		\item [3.] le Server S. reçoit la demande
		\item [4.] le Server S. déconnecte User S. du salon privé
		\item [5.] les clés du salon privé sont renouvelées
  		\item [6.] les nouvelles clés sont envoyées à tous les User S. connectés sur le salon privé
	 \end{itemize}
	 }
	{} %Flots d'exceptions

\subsection{UC 11 : Déconnexion Serveurs}

\fiche
	{Déconnexion Serveurs} %nom
	{Users / User S. / Server / Server S. / Admin} %acteurs concernés
	{Un utilisateur est déconnecté d’un serveur de chat après un problème technique ou de coupure réseaux} %description
	{Être connecté sur un des deux serveurs} %préconditions
	{L’utilisateur est déconnecté du serveur de chat (privé) :
	\begin{itemize}
\item si l’utilisateur était dans un salon privé, alors il y a un renouvellement des clés du salon privé
\item si l’utilisateur était l’administrateur d’un salon privé, alors le salon privé se ferme
	\end{itemize}} % événements déclenchants
	{} %conditions d'arret
	{\begin{itemize}  %flot d'événements (acteurs)
		\item [1.] Déconnexion après un problème technique
		\item[]  
	 \end{itemize}
	} 
	{\begin{itemize}  %flot d'événements (systeme)
		\item []
		\item [2.] événements système UC.8 ou UC.9
		\item [3.] si User S. connecté dans un salon privé $\rightarrow$ évènements système UC.10
		\item [4.] si Admin $\rightarrow$ UC.14
	 \end{itemize}
	 }
	{} %Flots d'exceptions


\subsection{UC 12 : Créer un salon privé}

\fiche
	{Créer un salon privé} %nom
	{User S. / Server S. / Client S / Server} %acteurs concernés
	{Un utilisateur sécurisé crée un salon privé sur le serveur de chat sécurisé} %description
	{L’utilisateur sécurisé est connecté sur le serveur de chat sécurisé} %préconditions
	{Création d’un salon privé, l’utilisateur sécurisé devient administrateur du salon privé} % événements déclenchants
	{} %conditions d'arret
	{\begin{itemize}  %flot d'événements (acteurs)
		\item [1.] L’utilisateur sécurisé clique sur la création d’un salon privé 
		\item[]  
		\item [3.] L’utilisateur saisit le nom du salon privé et la liste éventuelle des utilisateurs sécurisés invités
	 \end{itemize}
	} 
	{\begin{itemize}  %flot d'événements (systeme)
		\item []
		\item []
		\item [2.] le Client ouvre un formulaire de création d’un salon privé
		\item []
		\item []
		\item [4.] Création du salon privé dans Server et Server S
		\item [5.] Création des clés du salon privé par Server S.
		\item [6.] Changement de statut de l’utilisateur sécurisé en administrateur du salon privé et envoi d'une clef d'authentification
		\item [7.] Avertir les utilisateurs invités dans le salon privé qu’ils peuvent le rejoindre
		\item [8.] Ouverture du salon privé dans un nouvel onglet, côté administrateur, du salon privé
		\item [9.] Affichage du nom du salon privé dans la liste des salons privés sur la page d’accueil du serveur
	 \end{itemize}
	 }
	{RG.5} %Flots d'exceptions

\subsection{UC 13 : Rejoindre un salon privé}

\fiche
	{Rejoindre un salon privé} %nom
	{User S. / Server S.} %acteurs concernés
	{Un User S. rejoint un salon privé  } %description
	{Être connecté avec un compte utilisateur  sécurisé sur le serveur sécurisé.
Avoir reçu une invitation de la part d’un Admin à rejoindre le salon privé, ou avoir fait la demande de rejoindre un salon privé auprès d’un Admin} %préconditions
	{L’utilisateur rejoint le salon privé, modification des clés du salon et redistribution des clés aux utilisateurs présents dans le salon} % événements déclenchants
	{} %conditions d'arret
	{\begin{itemize}  %flot d'événements (acteurs)
		\item [1.] User S. accepte l’invitation à rejoindre un salon privé
		\item [2.] User S. rejoint le salon privé
	 \end{itemize}
	} 
	{\begin{itemize}  %flot d'événements (systeme)
		\item []
		\item []
		\item []
		\item [3.] Ajout du User S. dans le salon privé
		\item [4.] Renouvellement des clés du salon privé
		\item [5.] distribution des nouvelles clés du salon privé aux User S. connectés sur le salon privé
		\item [6.] Un message s’affiche sur le salon privé pour prévenir qu’un nouvel Utilisateur l’a rejoint
	 \end{itemize}
	 }
	{\begin{enumerate}
		\item User S. décline l’invitation à rejoindre le salon privé
		\item Admin refuse la demande de rejoindre le salon privé du User S. 
	\end{enumerate}} %Flots d'exceptions


\subsection{UC 14 : Fermeture du salon privé}

\fiche
	{Fermeture du salon privé} %nom
	{Admin / Server S. / Client S.} %acteurs concernés
	{L’Admin du salon privé ferme le salon privé} %description
	{Être Admin du salon privé} %préconditions
	{Fermeture du salon privé} % événements déclenchants
	{} %conditions d'arret
	{\begin{itemize}  %flot d'événements (acteurs)
		\item [1.] l’Admin ferme le salon privé
		\item[] 
		\item[] 
		\item[] 
		\item[] 
		\item [7.] Suppression des privilèges Admin au User S. ancien créateur du salon privé
	 \end{itemize}
	} 
	{\begin{itemize}  %flot d'événements (systeme)
		\item []
		\item [2.] le Client S. envoie la demande au Server S.
		\item [3.] le Server S. reçoit la demande
		\item [4.] Les User S. et Admin connectés sur le salon privé sont déconnectés.
		\item [5.] Suppression du salon privé sur le serveur de chat sécurisé et effacement des clés du salon
		\item [6.] suppression de la clé délivrée à l’admin lors de la création du salon 
	 \end{itemize}
	 }
	{Vérification que celui qui demande la fermeture du salon privé ait la clé (admin) généré lors de la création du salon. } %Flots d'exceptions

\subsection{UC 15 : Exclure utilisateur sécurisé d’un salon privé}

\fiche
	{Exclure utilisateur sécurisé d’un salon privé} %nom
	{Admin / Server S. / Client S.} %acteurs concernés
	{L’Admin du salon privé exclut un User S. connecté sur le salon privé} %description
	{Le User S. exclu doit être connecté sur le salon privé} %préconditions
	{User S. est exclu du salon, renouvellement des clés du salon privé} % événements déclenchants
	{} %conditions d'arret
	{\begin{itemize}  %flot d'événements (acteurs)
		\item [1.] l’Admin exclut le User S. du salon privé
	 \end{itemize}
	} 
	{\begin{itemize}  %flot d'événements (systeme)
		\item []
		\item [2.] le Client S. envoie la demande au Server S.
		\item [3.] le Server S. reçoit la demande
		\item [4.] le User S. exclu est déconnecté du salon privé par Server S.
		\item [5.] renouvellement des clés du salon privé, et distribution des nouvelles clés aux User S. connectés sur le salon privé
		\item [6.] Un message s’affiche dans le salon privé pour prévenir qu’un utilisateur a été exclu
	 \end{itemize}
	 }
	{} %Flots d'exceptions

\subsection{UC 16 : Invitation utilisateur sécurisé dans un salon privé}

\fiche
	{Invitation utilisateur sécurisé dans un salon privé} %nom
	{Admin / Server / User S. / Client S.} %acteurs concernés
	{L’Admin du salon privé invite un User S. à rejoindre le salon privé} %description
	{L’utilisateur invité doit être un utilisateur avec un compte  sécurisé} %préconditions
	{User S. reçoit une invitation à rejoindre le salon privé} % événements déclenchants
	{} %conditions d'arret
	{\begin{itemize}  %flot d'événements (acteurs)
		\item [1.] Admin envoie une invitation à un (des) User(s) S. pour rejoindre le salon privé
		\item []
		\item []
		\item [4.] le(s) User(s) S. reçoit(vent) l’invitation à rejoindre le salon privé sous forme d’un message
	 \end{itemize}
	} 
	{\begin{itemize}  %flot d'événements (systeme)
		\item []
		\item [2.] Le Client S. de l’admin envoie l’invitation au Server
		\item [3.] Le Server envoie l’invitation au Client S. du User S. invité dans le salon privé
	 \end{itemize}
	 }
	{} %Flots d'exceptions
	
\subsection{UC 17 : Envoi d’un message privé non sécurisé}

\fiche
	{Envoi d’un message privé non sécurisé} %nom
	{User / User S. / Client / Server} %acteurs concernés
	{Un User (S.) envoie un MP à un User (S.)} %description
	{Être connecté sur le même serveur de chat que le destinataire du message.} %préconditions
	{User (S.) reçoit un message privé dans un nouvel onglet :
		\begin{itemize}
			\item MP User vers User S.
			\item demander à User S si il accepte une communication non sécurisée, sinon fermer le MP.
			\item MP User S vers User
			\item prévenir User S. que la communication ne sera pas sécurisée	
		\end{itemize}
	} % événements déclenchants
	{} %conditions d'arret
	{\begin{itemize}  %flot d'événements (acteurs)
		\item [1.] User (S.) envoie un message privé à un autre User (S.) 
		\item []
		\item []
		\item [5.] le User (S.) qui reçoit le message est averti qu’un nouveau MP est reçu.
	 \end{itemize}
	} 
	{\begin{itemize}  %flot d'événements (systeme)
		\item []
		\item [2.] Le client envoie le message vers le serveur de chat
		\item [3.] le serveur de chat envoie le message au User (S.) destinataire
		\item [4.] Le client ouvre un nouvel onglet pour afficher le message privé
	 \end{itemize}
	 }
	{Refus de réception d’un message cas MP User vers User S.
	
	Refus d’envoie d’un MP cas User S. vers User} %Flots d'exceptions

\subsection{UC 18 : Envoi d’un message privé sécurisé}

\fiche
	{Invitation utilisateur sécurisé dans un salon privé} %nom
	{User S. / Client S. / Server} %acteurs concernés
	{Un User S. envoi un MP à un User S.} %description
	{Être connecté sur le même serveur de chat que le destinataire du message. } %préconditions
	{User S. reçoit un message privé dans un nouvel onglet} % événements déclenchants
	{} %conditions d'arret
	{\begin{itemize}  %flot d'événements (acteurs)
		\item [1.] User S. envoie un message privé à un autre User S.
		\item []
		\item []
		\item []
		\item []
		\item []
		\item []
		\item []
		\item [6.] le User S. qui reçoit le message est avertis qu’un nouveau MP est reçu.
	 \end{itemize}
	} 
	{\begin{itemize}  %flot d'événements (systeme)
		\item []
		\item [2.] Le Client S. chiffre le message et l’envoie vers le serveur de chat
		\item [3.] le serveur de chat envoie le message au client du User S. destinataire
		\item [4.] le Client S. du User S. destinataire déchiffre le message
		\item [5.] Le Client S. du User S. destinataire ouvre un nouvel onglet pour afficher le message privé
	 \end{itemize}
	 }
	{} %Flots d'exceptions

\subsection{UC 19 : Demande d’ajout dans un salon privé}

\fiche
	{Demande d’ajout dans un salon privé} %nom
	{User S. / Admin / Server / Client} %acteurs concernés
	{Un User S. fait la demande auprès d’un admin pour rejoindre un salon privé} %description
	{Avoir un compte sécurisé} %préconditions
	{L’admin reçoit la demande d’un User S. pour rejoindre le salon privé} % événements déclenchants
	{} %conditions d'arret
	{\begin{itemize}  %flot d'événements (acteurs)
		\item [1.] User S. fait la demande de rejoindre un salon privé auprès d’un Admin
		\item []
		\item []
		\item [4.] l’admin reçoit la demande du User S.
		\item [5.] l'admin accepte  la demande et envoie une invitation au User S.
	 \end{itemize}
	} 
	{\begin{itemize}  %flot d'événements (systeme)
		\item []
		\item [2.] Le Client du User S. envoie la demande au Server
		\item [3.] Le Server envoie la demande au Client de l’admin
		\item []
		\item []
		\item [6.] Server S. envoie l’invitation au User S.

	 \end{itemize}
	 }
	{Admin refuse la demande de rejoindre un server par User S.} %Flots d'exceptions

\section{Diagrammes de cas d'utilisation}

\subsection{UC 1}


\begin{tikzpicture}
\umlactor{User}
\umlactor[x=14, y=4]{AC}

\umlusecase[x=3, y=2, width=3cm]{Création compte sécurisé}
\umlusecase[x=10, y=4, width=3cm]{Suppression compte sécurisé}
\umlusecase[x=3, y=-1]{Rejoindre Server}

\umlusecase[x=7, y=1]{Quitter Server}
\umlusecase[x=10, width=3cm]{Envoi message privé}
\umlusecase[x=10, y=-2, width=3cm]{Envoi message au salon général}
\umlusecase[x=7, y=-3]{Déconnexion}

\umltrans{User}{usecase-1}
\umltrans{AC}{usecase-2}
\umltrans{User}{usecase-3}

\umlinclude{usecase-1}{usecase-2}

\umlextend{usecase-4}{usecase-3}
\umlextend{usecase-5}{usecase-3}
\umlextend{usecase-6}{usecase-3}
\umlextend{usecase-7}{usecase-3}
\end{tikzpicture}

\subsection{UC 2}

\begin{tikzpicture}
\umlactor[y=16]{User S}
\umlactor{Admin}

\umlusecase[x=4, y=16, width=2.5cm]{Rejoindre Server S} % 8
\umlusecase[x=6, y=20]{Quitter Server S}				% 9
\umlusecase[x=9, y=19]{Suppression compte}				% 10
\umlusecase[x=10, y=18]{Déconnexion}                    % 11
\umlusecase[x=10, y=16, width=3cm]{Envoi message privé sécurisé}    % 12
\umlusecase[x=9, y=14, width=3cm]{Envoi message sur salon général}			% 13
\umlusecase[x=7.5, y=12, width=3cm]{Rejoindre salon privé} % 14
\umlusecase[x=14, y=12, width=2.5cm]{Quitter salon privé} % 15
\umlusecase[x=2.5, y=12, width=2.5cm]{Demande rejoindre salon}					% 16
\umlusecase[x=5, y=6]{Créer salon privé}				% 17
\umlusecase[x=14, y=7]{Invitation User S}				% 18
\umlusecase[x=12,y=4,width=3cm]{Envoi de message dans salon} % 19
\umlusecase[x=10,y=2,width=3cm]{Fermeture du salon privé} % 20
\umlusecase[x=6]{Exclure User S}						%21

\umltrans{Admin}{User S}
\umltrans{User S}{usecase-8}
\umltrans{Admin}{usecase-18}
\umltrans{Admin}{usecase-19}
\umltrans{Admin}{usecase-20}
\umltrans{Admin}{usecase-21}

\umlextend{usecase-9}{usecase-8}
\umlextend{usecase-10}{usecase-8}
\umlextend{usecase-11}{usecase-8}
\umlextend{usecase-12}{usecase-8}
\umlextend{usecase-13}{usecase-8}
\umlextend{usecase-14}{usecase-8}
\umlextend{usecase-16}{usecase-8}
\umlextend{usecase-17}{usecase-8}

\umlextend{usecase-15}{usecase-14}
\umlextend{usecase-19}{usecase-14}

\umlinclude{usecase-18}{usecase-14}
\umlinclude{Admin}{usecase-17}

\end{tikzpicture}

\section{Exigences opérationnelles}
\begin{enumerate}
\item [EO.1] : Confidentialité, intégrité et authentification des messages (seuls les membres du salon peuvent voir les messages et vérifier leur authenticité).
\item [EO.2] : Création d’une interface utilisateur.
\item [EO.3] : Gérer la non répudiation des messages échangés (les certificats doivent pouvoir authentifier l'auteur du message).
\item [EO.4] : Autorité de certification (AC).
\item [EO.5] : Communication multi--salons privés.
\item [EO.6] : Protocoles cryptographiques n’impactant pas sur  la rapidité d’échange des messages.
\item [EO.7] : Entrée et sortie d’un utilisateur dans un salon rapides.
\end{enumerate}

\section{Exigences d'interface}
\begin{enumerate}
\item [EI.1] : Création d’un serveur sécurisé « proxy » utilisable sur des protocoles de chat existants.
\end{enumerate}

\section{Exigences de qualité}
\begin{enumerate}
\item [EQ.1] : Assurer la confidentialité, l’intégrité et l’authentification des messages échangés entre les utilisateurs.
\end{enumerate}

\section{Exigences de réalisation}
\begin{enumerate}
\item [ER.1] : Documentation détaillée sur l’utilisation du protocole sécurisé « proxy ».
\item [ER.2] : Documentation détaillée sur l’ensemble de l’application.
\item [ER.3] : Utilisation de openssl dans les algorithmes cryptographiques.
\item [ER.4] : Utilisation de tinyca pour la gestion des certificats.
\end{enumerate}


\end{document}