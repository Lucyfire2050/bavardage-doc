\documentclass[a4paper,11pt,french]{article}
\usepackage[utf8]{inputenc}

\usepackage[T1]{fontenc}
\usepackage[francais]{babel} 
\usepackage[top=2cm, bottom=2cm, left=2cm, right=2cm, includeheadfoot]{geometry} %pour les marges
\usepackage{lmodern}
\usepackage{pict2e}

\usepackage{fancyhdr} % Required for custom headers
\usepackage{lastpage} % Required to determine the last page for the footer
\usepackage{extramarks} % Required for headers and footers
\usepackage{graphicx} % Required to insert images
\usepackage{tabularx, longtable}
\usepackage{color, colortbl}

\linespread{1.1} % Line spacing

% Set up the header and footer
\pagestyle{fancy}
\lhead{\textbf{\hmwkClass -- \hmwkSubject \\ \hmwkTitle \\ \hmwkDocName}} % Top left header
\rhead{\includegraphics[width=10em]{logo_univ.png}}
\lfoot{\lastxmark} % Bottom left footer
\cfoot{} % Bottom center footer
\rfoot{Page\ \thepage\ / \pageref{LastPage}} % Bottom right footer
\renewcommand\headrulewidth{0.4pt} % Size of the header rule
\renewcommand\footrulewidth{0.4pt} % Size of the footer rule

\setlength{\headheight}{40pt}

\newcommand{\hmwkTitle}{Chat sécurisé} % Assignment title
\newcommand{\hmwkClass}{Master 1 SSI } % Course/class
\newcommand{\hmwkAuthorName}{Ambroise Ismael Kabore} % Your name
\newcommand{\hmwkSubject}{Conduite de projet} % Subject
\newcommand{\hmwkDocName}{Politique de certification} % Document name

\newcommand{\version}{0.2} % Document version
\newcommand{\docDate}{18 Décembre 2012} % Document date
\newcommand{\checked}{Julien Legras} % Checker name
\newcommand{\approved}{} % Approver name

\definecolor{gris}{rgb}{0.95, 0.95, 0.95}

\title{
\vspace{2in}
\textmd{\textbf{\hmwkClass :\ \hmwkTitle}}\\
\normalsize\vspace{0.1in}\small{Due\ on\ \hmwkDueDate}\\
\vspace{0.1in}\large{\textit{\hmwkClassInstructor\ \hmwkClassTime}}
\vspace{3in}
}

\author{\hmwkAuthorName}
\date{} % Insert date here if you want it to appear below your name


\begin{document}

\pagestyle{fancy}

\vspace*{5cm}
\begin{center}\textbf{\Huge{\hmwkDocName}}\end{center}
\vspace*{7cm}
	

\fcolorbox{black}{gris}{
\begin{minipage}{15cm}
\begin{tabularx}{10cm}{lXl}
	\bfseries{Version} & & \version\\
	& & \\
	\bfseries{Date} & & \docDate\\
	& & \\
	\bfseries{Rédigé par} & & \hmwkAuthorName \\
	& & \\
	\bfseries{Relu par} & & \checked \\
	& & \\
	\bfseries{Approuvé par} & & \approved \\
	& & \\
\end{tabularx}
\end{minipage}
}

\newpage

%Tableau de mises à jour
\vspace*{1cm}
\begin{center}
\textbf{\huge{MISES À JOUR}}\\
\vspace*{3cm}
	\begin{tabularx}{16cm}{|c|c|X|}
	\hline
	\bfseries{Version} & \bfseries{Date} & \bfseries{Modifications réalisées}\\
	\hline
	0.1 & 16/12/2012 & Création\\
	\hline
	0.2 & 18/12/2012 & Après relecture par Julien\\
	\hline
	\end{tabularx}
\end{center}

%La table des matières
\clearpage

\tableofcontents
\newpage

\section{Introduction}
\subsection{Présentation générale}

Le groupe F du Master professionnel 1 SSI (2012-2013), dans le cadre de son projet annuel doit étudier et réaliser un outil de chat sécurisé. L'objectif de ce projet est d'étudier les protocoles cryptographiques permettant a plusieurs utilisateurs de s'authentifier et de communiquer de manière sécurisée à travers un outil de messagerie instantanée.


\subsection{ Identification }

Ce document est la politique de certification du chat sécurisé Bavardage et est identifié par le nom Bavardage-pc.


\subsection{ Autorités, applications et groupes d'utilisateurs concernés }
\subsubsection{Autorité administrative (AA)}

Composante de l'IGC qui définit et fait appliquer les politiques de certification et les déclarations des pratiques de certification par la PKI, ainsi que la politique de sécurité générale de la PKI.
	
\subsubsection{Autorité de certification (AC)}
C'est l'autorité à laquelle les utilisateurs font confiance pour émettre et gérer des clés,des certificats, et des révocations. L'AC est gérée par l'administrateur du serveur sécurisé du chat.


\subsubsection{Autorité d'enregistrement (AE)}

Une autorité d’enregistrement est une composante de l’Infrastructure de Gestion des Clés qui vérifie les données propres au demandeur de certificat ainsi que les contraintes liées à l’usage d’un certificat, conformément à la politique de certification. Elle assure le lien entre l'Autorité de Certification et les utilisateurs. 

\subsubsection{Autorité de dépôt (AD)}
C'est l'autorité qui stocke les certificats numériques ainsi que les listes de révocation.

\subsubsection{ Utilisateur demandeur (UD)}

C'est la personne physique ayant directement par la loi ou par délégation, le pouvoir de demande de certificat portant le nom du chat et du signataire de la convention.

\subsubsection{ Utilisateur Final (UF)}

C'est la personne physique qui utilise les certificats. C'est en général les utilisateurs du chat.

\subsubsection{Utilisateur signataire convention (USC)}

C'est la personne physique ayant directement par la loi ou par délégation, le pouvoir de signer les conventions passées avec Bavardage. Tous les certificats porteront le nom du chat et le nom du signataire de la convention passée bavardage. 
		
\subsubsection{Type d'applications concerné}

L'usage des certificats doit permettre l' identification et l'authentification d'un utilisateur Bavardage décline toute responsabilité pour tout usage de certificat qui serait sans rapport avec le chat.

\subsection{Points de contact}
\subsubsection{Autorité de sécurité compétente dans l'accréditation des composants d'une PKI}
		À compléter
		
\subsubsection{Personnes à contacter concernant ce document
Sans objet}		

\subsubsection{Personnes habilitées à déterminer la conformité de la DPC avec la politique de certification}

Les personnes habilitées à déterminer la conformité de la DPC avec la politique de certification sont nommées par l’Autorité Administrative(AA).

\section{Dispositions d'ordre générale}
\subsection{Obligations}
Les obligations suivantes sont communes à toutes les composantes de la PKI:
Protéger sa clé privée et ses données d’activation en intégrité et en confidentialité.
N’utiliser ses clés publiques et privées qu’aux fins pour lesquelles elles ont été émises et avec les outils spécifiés, en vertu de la présente politique.
Respecter et appliquer la PC.
Mettre en œuvre les moyens (techniques et humains) nécessaires à la réalisation des prestations auxquelles l’entité concernée s’engage.

\subsubsection{Obligations des acteurs}
\begin{tabular}{|l|p{10cm}|}
  \hline
  Autorité Administrateur &  
  \begin{itemize}
   \item Valider la politique de certification
   \item Établir la conformité entre la PC et la DPC
   \end{itemize}

 \\
  \hline
  Autorité d'enregistrement & 
  \begin{itemize}
  \item  Respecter la législation relative au respect des données
d’identification personnelles
\item Publier un formulaire de demande de certification.
\item Vérifier l'exactitude des mentions qui établissent l'identité du
demandeur et de l'entreprise.
\item Si elle est saisie d'une demande de révocation, elle doit en vérifier l'origine et l'exactitude.
\item Traiter les demandes de certificat.
\item Conserve et protège en confidentialité et intégrité toutes les données collectées lors de la demande de certificat.
\item Doit se soumettre à tout contrôle technique et audits que pourrait demander l'AA.
  \end{itemize}
  \\
  \hline
  Autorité de certification &
  \begin{itemize}

\item S'engager à diffuser publiquement la politique de certification, la liste des certificats révoqués, et la liste des certificats auxquels la clé racine de l'IGC est subordonnée.
\item Documenter les schémas de certification qu’elle entretient avec
d’autres AC ou d’autres PKI.
\item Respecter le résultat d’un contrôle de conformité et remédier aux non conformités qu’il révèle
\item Documenter ses procédures internes de fonctionnement.
\item Respecter la PC et appliquer la DPC.
\item Doit se soumettre à tout contrôle technique et audits que pourrait demander l'AA.

  \end{itemize}
 \\
 \hline
 Utilisateur Demandeur &
 \begin{itemize}
 \item Il doit respecter la convention avec le chat bavardage et la PC
\item Il doit procéder sans délais aux révocations en cas de perte ou
compromission des clés privées. 
 
 \end{itemize} 
  
 \\
 \hline
 Utilisateur Final &
 \begin{itemize}
 \item Se conformer aux règles de la présente politique de certification.
\item Respecter les conditions d'utilisation des clés et certificats, et protéger ceux-ci.
\item A défaut de remplir cette obligation il assume seul tous les risques de ses actions non conformes aux exigences de la présente politique.
 
 \end{itemize}
\\
\hline
  
\end{tabular}
\subsubsection{Responsabilité des acteurs}

\begin{tabular}{|l|p{10cm}|}
  \hline
  Autorité Administrateur &  
  \begin{itemize}
  \item Résolution des litiges
   \end{itemize}

 \\
  \hline
  Autorité d'enregistrement & 
  \begin{itemize}
\item Enregistrement d’une demande en provenance des utilisateurs (UD).
\item Conservation et protection en confidentialité et en intégrité des données personnelles d’identification transmises pour l’enregistrement.
\item Seule l'AC peut mettre en cause la responsabilité de l'AE, ce qui exclut explicitement tout engagement de l'AE envers les utilisateurs demandeurs ou finaux.

  \end{itemize}
  \\
  \hline
  Autorité de certification &
  \begin{itemize}
\item Génération des certificats dans le cadre des procédures définies dans la PC et DPC.
\item Assure la responsabilité du service de publication, elle est responsable de l'information des utilisateurs des procédures à suivre tout au long du cycle de vie des certificats.


  \end{itemize}
 \\
 \hline
 Utilisateur Demandeur &
 \begin{itemize}
\item Il réalise les opérations de demande et révocation de certificats.
 
 \end{itemize} 
  
 \\
 \hline
 Utilisateur Final &
 \begin{itemize}
\item Il est responsable de la sécurité de son poste de travail.
 
 \end{itemize}
\\
\hline
  
\end{tabular}             

\subsubsection{Interpretation de la loi}
\begin{itemize}
\item Loi
Suivant la législation nationale :
Les données à caractère personnel d’une personne physique doivent être protégées suivant la loi ... À compléter
\item Résolution de litiges
Sans objet
\end{itemize}

\subsubsection{Publication et Services associés}
\begin{itemize}

\item Publication d’informations sur la PKI

Les informations concernant la PKI publiées sont les suivantes :
\begin{itemize}
\item La Politique de certification (PC)
\item La liste des certificats révoqués(LCR)
\end{itemize}

\item Fréquence de publication
\\Les informations seront publiées suivant un temps T

\item Service de publication
\\L' AC rend un service de publication des certificats qui se matérialisera par un annuaire. Ce service de publication met à disposition des utilisateurs suivant TEMPS=TempDispoPub les informations suivantes :
\begin{itemize}
\item les informations concernant la PKI
\item la liste des certificats révoqués
\end{itemize}

\end{itemize}

\subsubsection{Contrôle de conformité}
À compléter

\subsubsection{Politique de confidentialité}
\begin{itemize}
\item Type d'informations considérées comme confidentielles

Cas des informations confidentielles à caractère secret (obligation de discrétion). Il s’agit d’informations nécessaires au bon fonctionnement de la PKI :
\begin{itemize}
\item clés privées propres à la composante concernée
\item Politique de Certification 
\item Archives des conventions cryptographiques
\item La DPC
\\
\end{itemize}



Cas des informations confidentielles à caractère privatif (exigence de séclusion). Il s’agit d’informations nécessaires à l’opérabilité de l’IGC :
données personnelles d’identification nominative d’un utilisateur demandeur (identifiants nominatifs), les utilisateurs demandeurs disposent d'un droit d'accés, de rectification et d'opposition à la cession de toute information les concernant.
\\

\item Type d'informations considérés comme non confidentielles
\\Les informations publiées sur la PKI (cf. 2.1.4) sont considérées comme non confidentielles.
\\
\item Delivrance aux autorités légales
\\les autorités légales peuvent réclamer l’identité de l’individu ou de l’organisme qu’il représente. En aucun cas, le recouvrement de clé de signature ni de clé de certification ne doit être effectué.
\\

\item Delivrance à la demande du propriétaire
\\Les informations relatives à un utilisateur final et définies comme confidentielles en 2.1.6 ne peuvent être divulguées qu’à leur propriétaire (demandeur du certificat).
\end{itemize}

\section{Identification et authentification}
\subsection{Enregistrement initial}
\subsubsection{Convention de noms}
Le nom de l'Abonné figure dans le champ "Nom de l'organisation" du certificat au format X.509. Cette mention est obligatoire. Il est constitué du prénom usuel et du nom patronymique. Ce nom est celui de l'Abonné tel qu'il figure dans les documents d'État Civil.

\subsubsection{Nécessité d'utilisation de noms explicites}

Les informations portées dans les champs du certificat CA Certificat sont décrites ci-dessous de manière explicite :
\begin{itemize}
\item Pays : le code du pays sur 2 lettres
\item Mot de passe : requis pour la signature
\item Password : confirmation du mot de passe
\item Etat ou nom de province : 
\item Localité : la ville
\item Nom de l'organisation: Nom de l'entreprise ou de la personne
\item Unité organisationnelle : departement
\item Adresse eMail : Adresse électronique de l'abonné

\end{itemize}

\subsubsection{Règles d'interprétation des différentes formes de nom}
Ces informations sont établies par le groupe de projet et reposent essentiellement sur les règles
suivantes :
\begin{itemize}
\item Tous les caractères sont sans accents ni caractères spécifiques à la langue française;
\item les prénoms et noms composés sont séparés par des tirets " - ".

\end{itemize}

\subsubsection{Unicité des noms}
L’unicité d’un certificat est basée sur l’unicité de son numéro de série au sein de l'AC.

\subsubsection{Procédures de résolution des litiges sur la révendication d'un nom}
Lorsque le nom à inclure dans un certificat provoque un litige avec un autre utilisateur, l’autorité d’enregistrement à qui la demande de certification a été formulée proposera une procédure de résolution amiable du litige.

\subsubsection{Preuve de la possession d'une clé privée}
Génération du bi-clé par le demandeur et utilisation d'un protocole adapté.

\subsubsection{Authentification de l'identité d'un individu}
Seul l'utilisateur demandeur (UD) est authentifié. L'authentification de l' UD est réalisée lors de la procédure de connexion au serveur sécurisé du chat.

\subsection{Re-génération de certificat}
Les certificats sont à renouveler suivant un temps T = temps de validité du certificat. L'utilisateur demandeur refait une nouvelle demande de certificat. Un certificat révoqué ne peut pas être regénéré.
\subsection{Authentification d'une demande de révocation}
L'authentification d'une demande de révocation est effectuée par l'Autorité d'Enregistrement.
Sont demandés :
\begin{itemize}
\item le nom du demandeur,
\item le prénom du demandeur,
\item le motif de révocation,
\item l’adresse électronique de l'Abonné.
\end{itemize}

\section{Besoins opérationnels}

\subsection{Demande de certificat}

Elle se déroule en deux temps :
\begin{itemize}
\item L'utilisateur demandeur remplit le formulaire de demande de création d'un compte sécurisé.
\item Après réception du formulaire, l'AC traite la demande, génère le certicat et l'envoi à l'utilisateur demandeur qui devient un utilisateur final.
 \end{itemize}
 
\subsection{Génération de certificat}
\begin{itemize}
\item l'UD transmet une demande de certificat conformément au 4.1. \item L'autorité d'enegistrement vérifie la validité des informations portées par le formulaire de demande
\item Si le formulaire est valide, il est transmis à l'AC qui génère le certificat, le transmet à l'utilisateur demandeur et le stocke dans l'autorité de dépôt.
\item En cas de non validité des informations un message avec le motif du rejet est envoyé à l'utilisateur demandeur.

\end{itemize}

\subsection{Acceptation d'un certificat}
Sans réponse à l'envoi du certificat à l'UD. Envoyé par l'AC, il est considéré comme accepté par l'UD qui reconnaît de fait les termes et les conditions d'utilisations et assume les responsabilités liées à son utilisation. L'acceptation d'un certificat vaut acceptation de la PC en référence.

\subsection{Suspension et révocation d'un certificat}
\subsubsection{Causes possibles de révocation d'un certificat}

Les causes de révocation du certificat d'une AC peuvent être :
\begin{itemize}
\item Compromission, suspicion de compromission, vol, perte de certificat.
\item Compromission de clé publique d’une AC
\item Compromission, suspicion de compromission, vol, perte de la clé privée d’une AC
\item Non respect de la politique de certification ou de la déclaration des pratiques de certification.
\item Décision suite à un contrôle de conformité.
\item Cessation d’activité de l’AC.
\\

Les causes possibles de révocation du certificat d’un utilisateur final sont les suivantes :
\item Compromission, suspicion de compromission, vol ou perte de la clé privée de l’utilisateur final.
\item Compromission, suspicion de compromission, vol ou perte du certificat de l’utilisateur final
\item Compromission de clé publique de l’utilisateur final.
\item Révocation du certificat de l’AC émettrice du certificat.
\item Non respect du contrat ou de la convention liant un utilisateur final à la PKI.
\item Changement d’informations contenues dans le certificat (changement de fonctions de l’utilisateur, changement de nom, etc.)

\end{itemize}

\subsubsection{Publication des causes de révocation}
Elles ne sont pas publiées.

\subsubsection{Contrôle de la Liste de révocation}
Les utilisateurs finaux sont autorisés à consulter la liste de révocation.

\subsubsection{Personnes abilitées à demander une révocation}
Les personnes abilitées à demander une révocation sont:
\begin{itemize}
\item L'AC
\item L'AE
\item L'AA
\item Le serveur de chat sécurisé
\end{itemize}

\subsubsection{Procédure de demande de révocation}
La demande de révocation se fait soit
en envoyant un mail à l'AA, soit en remplissant un formulaire qui sera envoyé à l'AE.

\subsubsection{Temps de traitement d'une révocation}
Les demandes de révocation doivent être traitées suivant un temps T = temps de révocation.

\subsubsection{Fréquence de mise à jour de la liste des certificats révoqués}
L'AC garantit aux utilisateurs de ses certificats la mise à disposition d'une liste de certificats révoqués à jour
suivant un temps T = fréquence mise à jour liste de révocation.

\subsection{Journalisation}
\subsubsection{Types d'évènements enregistrés}
Les opérations réalisées sur la PKI seront enregistrées.

\subsubsection{Fréquence de traitement des journaux d'évènement}
Le processus de journalisation enregistre en temps réel les opérations effectuées, le contournement du processus n'est pas possible.

\subsubsection{Durée de retention d'un journal d'évènements}
Les journaux doivent être conservés pour une période
minimale T = temps retention journal d'évènements.

\subsubsection{Copie de sauvegarde des journaux d'évènements}
Des copies de sauvegardes des journaux d'évènement doivent être faites suivant un temps T = temps sauvegarde journaux d'évènements. Les archives de journaux d'événements sont protégés au même niveau que les journaux d'événements originaux.

\subsubsection{Imputabilité}
L’imputabilité d’une action revient à la personne, ou le système l’ayant exécutée et dont le nom figure dans le champ « nom de l’exécutant » du journal d’événements.

\subsection{Archives}
\subsubsection{Type de données à archiver}
Les données à archiver sont au moins les suivantes :
\begin{itemize}
\item Certificats d’utilisateurs (2 ans).
\item Liste de révocation (2 ans).
\item Données relatives à la demande de certificats (2 ans).
\item Les notifications (messages, etc) (2 ans).
\item Les journaux d'évènements (2 ans).
\end{itemize}

\subsubsection{Période de rétention des archives}
Les archives sont conservées pendant un temps T = temps conservation archive.

\section{Contrôle de sécurité physique et des procédures}
\subsection{Contrôle physique}
Pas concerné (Nous ne construirons pas de sites pour la PKI).
\subsection{Contrôle des procédures}
\subsubsection{Rôle}
On distingue un unique administrateur.

\subsubsection{Nombre de personnes nécessaire à chaque tâche}
Pas de spécification

\subsubsection{Identification et authentification}
La connexion d'un exploitant à la PKI nécessite son identification, identification à laquelle est associée son rôle au sein de la PKI.

\section{Contrôles techniques de sécurité}

\subsection{Génération et Délivrance de clé}
\subsubsection{Génération de bi-clé}
L'UD génère lui-même sa bi-clé. L'AC décline toute responsabilité pour une utilisation autre que celle définie dans la PC, y
compris pour l'authentification et l'identification mutuelle de deux UF. 
\subsubsection{Génération de clé privée}
Pas concerné(Le client de chat sécurisé génère les clés privées dans notre cas).

\subsubsection{Délivrance de clé privée}
Pas concerné(Le client de chat sécurisé génère les clés privées dans notre cas).

\subsection{Protection de clé privée}
Pas concerné (Le client de chat sécurisé génère les clés privées dans notre cas).

\section{Profils des certificats et listes des certificats révoqués}
\subsection{Profil de certificat}
Les certificats utilisés sont les certificats X.509 v3.\\

\begin{tabular}{|l|p{10cm}|}
\hline
Version  & Numéro de version du certificat 
\\
\hline
Serial number  & Numéro de série du certificat
\\
\hline
signature  & Identifiant de l'algorithme de signature de l'AC 
\\
\hline
Issuer  & Nom de l'AC
\\
\hline
Validity Period  & Période de validité
\\
\hline
Subject  & Nom de l'entité
\\
\hline
Subject public key info &  Identifiant de l'algorithme d'usage de la clé publique contenue dans le certificat, et valeur de la clé publique
\\
\hline
 Issuer unique identifier  & Identification unique de l'AC
\\
\hline
Subject unique identifier  & Identifiant unique de l'entité
\\
\hline
Extensions  & Les extensions sont soit standardisées, soit à la discrétion de l’autorité de certification
\\
\hline
\end{tabular}


\subsection{Profil de liste de révocation}


\begin{tabular}{|l|p{10cm}|}
\hline
Version  & Numéro de version de la liste de révocation
\\
\hline
Signature  & Identifiant de l'algorithme de signature de l'AC

\\
\hline
Issuer  & Nom de l'AC qui signe les certificats
 
\\
\hline
ThisUpdate  & Date de génération de la liste de révocation
\\
\hline
NextUpdate  & Prochaine date à laquelle cette Liste de révocation sera mise à jour 

\\
\hline
RevokedCertificates  & liLte des numéros de série des certificats révoqués, contenant les champs suivants :
\begin{itemize}
\item userCertificate : Numéro de série du certificat révoqué 
\item revocationDate : Date à laquelle le certificat a été révoqué

\end{itemize}

\\
\hline
CrlExtensions &  liste des extensions de la LCR :
\begin{itemize}
\item authorityKeyIdentifier : identifiant de la clé publique de l’AC qui a signé la liste de révocation
\item CRLNumber : numéro de série de la liste de révocation 
\end{itemize}
\\
\hline

\end{tabular}

\section{Administration et spécification}
\subsection{Procédure de modification de ces spécifications}
Toute modification jugée par l’administrateur de l’AC comme pouvant entraîner une perte de la conformité d’un certificat avec la politique de certification ou avec la DPC doit être
approuvée par l’Autorité administrative. 

\subsection{Politiques de publication et de notification}
l'AC avertit les UF des modifications apportées aux spécifications par courrier électronique.

\subsection{Procédures d'aprobation des DPC}
L’approbation d’une DPC est confiée à l’Autorité administrative qui vérifie l’adéquation de la DPC fournie avec la politique de certification.

\newpage
\textbf {Acronymes}


\begin{tabular}{|l|p{10cm}|}
\hline
AA  & Autorité Administrative
\\
\hline
AC & Autorité de Certification
\\
\hline
AE  & Autorité d'Enregistrement 
\\
\hline
AD  & Autorité de Dépôt
\\
\hline
UD  & Utilisateur Demandeur 
\\
\hline
UF & Utilisateur Final
\\
\hline
USC &  Utilisateur Signataire Convention
\\
\hline
PC & Politique de Certification
\\
\hline
DPC & Déclaration des Pratiques de Certification
\\
\hline
PKI & Public Key Infrastructure
\\
\hline
IGC & Infrastructure de Gestion de clés
\\
\hline
LCR & Liste de Certificats Révoqués
\\
\hline

\end{tabular}

































































\end{document}
