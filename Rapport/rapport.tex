\documentclass[a4paper,11pt,french]{book}
\usepackage[utf8]{inputenc}

\usepackage[T1]{fontenc}
\usepackage[francais]{babel} 
\usepackage[top=2cm, bottom=2cm, left=2cm, right=2cm, includeheadfoot]{geometry} %pour les marges
\usepackage{lmodern}
\usepackage{pict2e}
\usepackage{tikz}	
%\usepackage{tikz-uml}
\usepackage{fancyhdr} % Required for custom headers
\usepackage{lastpage} % Required to determine the last page for the footer
\usepackage{extramarks} % Required for headers and footers
\usepackage{graphicx} % Required to insert images
\usepackage{tabularx, longtable}
\usepackage{color, colortbl}
\usepackage{lscape}
\usepackage[hidelinks]{hyperref}
\usepackage{longtable}
\usepackage{multirow}
\usepackage{rotating}
\usepackage{pgfgantt}
\usepackage{gensymb}
\usepackage[toc,page]{appendix} 

\usepgflibrary{arrows} % for pgf-umlsd

\usetikzlibrary{trees,shapes.geometric,arrows,decorations.pathmorphing,backgrounds,fit,positioning,shapes.symbols,chains	}

\linespread{1.1} % Line spacing

% Set up the header and footer
\pagestyle{fancy}
\lhead{\textbf{\hmwkClass -- \hmwkSubject \\ \hmwkTitle \\ \hmwkDocName}} % Top left header
\rhead{\includegraphics[width=10em]{logo_univ.png}}
\lfoot{\lastxmark} % Bottom left footer
\cfoot{} % Bottom center footer
\rfoot{Page\ \thepage\ } % Bottom right footer
\renewcommand\headrulewidth{0.4pt} % Size of the header rule
\renewcommand\footrulewidth{0.4pt} % Size of the footer rule

\setlength{\headheight}{40pt}

\newcommand{\hmwkTitle}{Chat sécurisé} % Assignment title
\newcommand{\hmwkClass}{Master 1 SSI } % Course/class
\newcommand{\hmwkAuthorName}{Charles Ango, Ismael Kabore,  Julien Legras, Yves Nouafo, Jean-Baptiste Souchal} % Your name
\newcommand{\hmwkSubject}{Conduite de projet} % Subject
\newcommand{\hmwkDocName}{Rapport de projet} % Document name

\newcommand{\version}{1.2} % Document version
\newcommand{\docDate}{24 mai 2013} % Document date
\newcommand{\checked}{Magali Bardet} % Checker name
\newcommand{\approved}{} % Approver name

\definecolor{gris}{rgb}{0.95, 0.95, 0.95}

\title{
\vspace{2in}
\textmd{\textbf{\hmwkClass :\ \hmwkTitle}}\\
\normalsize\vspace{0.1in}\small{Due\ on\ \hmwkDueDate}\\
\vspace{0.1in}\large{\textit{\hmwkClassInstructor\ \hmwkClassTime}}
\vspace{3in}
}

\author{\hmwkAuthorName}
\date{} % Insert date here if you want it to appear below your name

\makeatletter
\def\chapter{\if@openright\cleardoublepage\else\clearpage\fi
  \global\@topnum\z@
  \@afterindentfalse
  \secdef\@chapter\@schapter}
\makeatother


\begin{document}
\pagestyle{empty}

\vspace*{1cm}
\begin{center}
\includegraphics[width=20em]{logo_univ.png}
\end{center}
\vspace*{2cm}
\begin{center}\textbf{\huge{Master 1 Sécurité des systèmes informatiques}}\end{center}
\vspace*{1cm}
\begin{center}
\textbf{\Huge{\hmwkDocName}: \hmwkTitle}
\end{center}

\vspace*{3cm}
	

\fcolorbox{black}{gris}{
\begin{minipage}{15cm}
\begin{tabularx}{10cm}{lXp{8cm}}
	& & \\
	\bfseries{Rédigé par} & & \hmwkAuthorName \\
	& & \\
	\bfseries{À l'attention de} & & \checked \\
	& & \\
	\bfseries{Date de rendu} & & \docDate\\
	& & \\
\end{tabularx}
\end{minipage}
}

\newpage


%%%%%%%%%%%%%%%%%%%%%%%%%%%%%%%%%%%%%%%%%%%%
% Introduction
% JB
%%%%%%%%%%%%%%%%%%%%%%%%%%%%%%%%%%%%%%%%%%%%
\frontmatter
\pagestyle{fancy}
\chapter{Introduction}


%La table des matières
\tableofcontents



\newpage
\mainmatter
%%%%%%%%%%%%%%%%%%%%%%%%%%%%%%%%%%%%%%%%%%%%
% PRÉSENTATION DU PROJET 
% Yves & Julien
%%%%%%%%%%%%%%%%%%%%%%%%%%%%%%%%%%%%%%%%%%%%
\chapter{Présentation du projet}
\section{Besoins du client}

\section{Solution proposée}

\section{Résultat}

\section{Problèmes rencontrés}




\newpage
%%%%%%%%%%%%%%%%%%%%%%%%%%%%%%%%%%%%%%%%%%%%
% MANUEL D'UTILISATION 
% Charles & Ismaël
%%%%%%%%%%%%%%%%%%%%%%%%%%%%%%%%%%%%%%%%%%%%
\chapter{Manuel d'utilisation}
\section{Récupération du projet}
 Les sources du projet sont disponibles sur un dépôt git. Elles peuvent être récupérées en ligne de commande ou en ligne. 
 
 \subsection{En ligne de commande}
 Placez vous dans un terminal et exécutez la commande suivante :
\begin{verbatim} 
    $git clone git://github.com/legrajul/bavardage.git 
\end{verbatim}

\subsection{En ligne}
 Une archive contenant le projet peut être téléchargé à l'adresse :

\section{Compilation}
\subsection{Dépendances Ubuntu}
Pour la compilation du projet il faut d'abord vérifier si toutes les dépendances sont satisfaites :
\begin{itemize}
\item cmake : permet de compiler un projet pour différentes plateformes.
\item valac-0.18 : compilateur Vala qui traduit le code source vala en code source C.
\item libgtk-3-dev : outil multi plateformes pour créer des interfaces graphiques.
\item libgee-dev : librairies de collections fournissant des classes basées sur GObject.
\item libglib2.0-dev : fichiers de développement pour la bibliothèque GLib.
\item libssl-dev : bibliothèque de développement SSL.
\item libsqlite3-dev : bibliothèque de développement SQLite.

\end{itemize}

\subsection{Compilation des sources}

Les commandes suivantes doivent être exécuter dans le dossier du projet git bavardage : 
\begin{verbatim}
    $ mkdir src/build 
    $ cd src/build 
    $ cmake .. 
    $ make
\end{verbatim}

\section{Exécution}



\newpage
%%%%%%%%%%%%%%%%%%%%%%%%%%%%%%%%%%%%%%%%%%%%
% CONCLUSION 
% JB
%%%%%%%%%%%%%%%%%%%%%%%%%%%%%%%%%%%%%%%%%%%%


\newpage
\appendix
\chapter{Documents de gestion de projet}
\chapter{Déclaration des pratiques de certification}
\chapter{Politique de certification}
\end{document}