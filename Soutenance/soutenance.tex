\documentclass{beamer}
\usepackage[utf8]{inputenc}
\usepackage[T1]{fontenc}
\usepackage[francais]{babel} 
\usepackage{lmodern}
\usepackage{hyperref}
\usepackage{tikz}
\usepackage{pgfgantt}
\usepackage{pgfcalendar}
\usepackage{ifthen}
\usepackage{etex}

\usetikzlibrary{trees,shapes.geometric,arrows,decorations.pathmorphing,backgrounds,fit,positioning,shapes.symbols,chains	}
 \tikzset{
    %Define standard arrow tip
    >=stealth',
    %Define style for boxes
    punkt/.style={
           rectangle, dashed,
           rounded corners,
           draw=black, very thin,
           minimum height=2em,
           minimum width = 2cm,
           text centered},
    square/.style={
           rectangle,
           draw=black, thick,
           minimum height=.5cm,
           minimum width = 1cm,
           text centered},
    data/.style={
           rectangle,
           draw=black, thick,
           minimum height= 2cm,
           minimum width = 2cm,
           text centered},
    % Define arrow style
    pil/.style={
           ->,
           thick,
           shorten <=1pt,
           shorten >=1pt,},
    asym/.style={
           <->,
           thin,
           shorten <=1pt,
           shorten >=1pt,
           red!100},
    sym/.style={
           <->,
           thin,
           shorten <=1pt,
           shorten >=1pt,
           blue!100}
}
\usepackage{pgfplots}
\usepackage{eurosym}
\usepackage{rotating}
\usepackage{array}
\usepackage{tikz-uml}
\usepackage{color, colortbl}
\usetheme{Antibes}
\usecolortheme{beaver}
\setbeamertemplate{sections/subsections in toc}[square]
\setbeamertemplate{blocks}[square]%

\author{Charles Ango - Ismaël Kabore - Julien Legras - Yves Nouafo - Jean-Baptiste Souchal}
\title{Soutenance projet annuel - Chat sécurisé}
\titlegraphic{\includegraphics[height=3em]{logo_univ.png}}
\institute{Master 1 Sécurité des Systèmes Informatiques}

\date{31/05/2013}

\begin{document}
\newcount\startdate
\pgfcalendardatetojulian{2013-01-021}{\startdate}

\protected\def\zzz{\ {\pgfcalendarjuliantodate{\numexpr\startdate\relax}{\year}{\month}{\day}\day/\month\global\advance\startdate1}}

{
\setbeamertemplate{headline}[default] 
\begin{frame}
  \titlepage
\end{frame}
}

\frame{
\frametitle{Sommaire}
\tableofcontents
}

\section{Présentation du projet}
\frame{
\frametitle{Présentation du projet}
\begin{block}{Sujet proposé par Magali Bardet}
	Réaliser un logiciel de messagerie instantanée \textbf{sécurisée} : client et serveur.\\
	S'inspirer des fonctionnalités d'IRC (Internet Relay Chat).
	Fonctionnalités demandées :
	\begin{itemize}
		\item gestion de la création et la suppression d'un compte sécurisé;
		\item création par un utilisateur d'une salle de discussion pour un groupe de personnes ;
		\item ajout/suppression d'un utilisateur autorisé dans une salle;
		\item assurance de confidentialité, d'intégrité et d'authentification sur les messages échangés;
		\item non-répudiation des messages.
	\end{itemize}
\end{block}
}

\section{Déroulement des sprints}
\subsection{Sprint 1}
\frame{
\frametitle[allowframebreaks=1.0]{Sprint 1}
\begin{block}{Tâches}
Retard cumulé : 1 jour

Raison(s): prise en main de SQLite
\end{block}

\begin{exampleblock}{Livraison}
Tests : OK

Délai respecté  (15 février 2013)

Réunion avec Mme Bardet post livraison le 19 février pour validation
\end{exampleblock}
}


\subsection{Sprint 2}
\frame{
\frametitle[allowframebreaks=1.0]{Sprint 2}

\begin{exampleblock}{Livraison}
Tests: OK

Délai respecté (15 mars 2013) : livraison d'une machine virtuelle

$\hookrightarrow$ mise à disposition d'une machine par Mr Macadré (configurée lors du sprint 3)

Réunion avec Mme Bardet post livraison le 3 avril pour validation
\end{exampleblock}

}

\subsection{Sprint 3}
\frame{
\frametitle[allowframebreaks=1.0]{Sprint 3}

\begin{block}{Tâches}
Retard : 3 semaines

Raison(s): contrôles continus pendant 2 semaines + 1 semaine \emph{off}
\end{block}

\begin{exampleblock}{Livraison}
Tests: OK

Délai réajusté avec Mme Bardet (3 mai 2013)

Réunions avec Mme Bardet pré livraison le 29 avril et post livraison le 22 mai
\end{exampleblock}

}

\section{Technique}
\frame{
\frametitle{Schéma global}

\begin{center}
\begin{tikzpicture}[node distance=-.01cm,font=\tiny,scale=1.05,every node/.style={transform shape}]
		\node[square, text width=1.5cm, fill=white!100] (appclient) at (0,0) {Application client};
		\node[square, text width=2cm, fill=white!100] (modclientsec) at (4,0) {Module client sécurisé};
		\begin{pgfonlayer}{background} 
		\node[punkt, fit=(appclient)(modclientsec), fill=blue!20] (groupclient) {};
		\end{pgfonlayer}
		
		\node[square, text width=1cm, fill=white!100] (PKI) at (2.5,-1.5) {PKI};
		\node[square, text width=1cm, fill=white!100] (serveur) at (0,-2.5) {Serveur};
		\node[square, text width=1.5cm, fill=white!100] (serveurs) at (4,-2.5) {Serveur sécurisé};
		\begin{pgfonlayer}{background} 
		\node[punkt, fit=(PKI)(serveur)(serveurs), fill=green!20] (groupserveur) {};
		\end{pgfonlayer}

		\node[text width=.7cm] (imgbd1) at (0,-4) {\includegraphics[height=3em]{computer-database.png}};
		\node[square, text width=.5cm, below=of imgbd1, fill=white!100] (bd1){BDD};
		\node[text width=.7cm] (imgfichiers) at (2.5,-4) {\includegraphics[height=3em]{computer-database.png}};
		\node[square, text width=.7cm, below=of imgfichiers, fill=white!100] (fichiers){BDD};
		\node[text width=.7cm] (imgbd2) at (4,-4) {\includegraphics[height=3em]{computer-database.png}};
		\node[square, text width=.5cm, below=of imgbd2, fill=white!100] (bd2){BDD};

		\begin{pgfonlayer}{background} 
		\node[punkt, fit=(imgbd1) (bd1) (imgbd2) (bd2) (imgfichiers) (fichiers), fill=orange!20] (stock) {};
		\end{pgfonlayer}


		\draw (appclient.east) edge[<->] (modclientsec.west);
		\draw (appclient.south) edge[<->] (serveur.north);
		\draw (modclientsec.south) edge[<->] (serveurs.north);
		
		\draw (serveur.south) edge[<->] (imgbd1.north);
		\draw (serveurs.south) edge[<->] (imgbd2.north);
		\draw (PKI.south) edge[<->] (imgfichiers.north);
		\draw (PKI.south east) edge[<->] (serveurs.north);
		
		\draw (PKI.west) edge[->, loop left = 90] (PKI.west);

		\begin{pgfonlayer}{background} 
		\node[minimum width=2.5cm, minimum height=6.3cm,rectangle, draw=black!60!green, very thick, fit=(appclient) (serveur) (imgbd1) (bd1)] (groupeclassic) {};
		\node[minimum width=4cm, minimum height=6.3cm,rectangle, draw=black!30!red, very thick, fit=(modclientsec) (PKI) (serveurs) (imgbd2) (bd2) (imgfichiers) (fichiers)] (groupesec) {};
		\end{pgfonlayer}
		
		\node[below=of groupeclassic, color=black!60!green] (legendgpeclassic) {Chat classique};		
		\node[below=of groupesec, color=black!30!red] (legendgpesec) {Partie sécurisée};

		\node[right=of modclientsec, minimum width=3cm] (lgdappcl) {Application client};
		
		\node[right=of groupserveur, minimum width=3cm] (lgsrv) {Serveurs};
		
		\node[right=of stock, minimum width=4cm, anchor=west] (lgbdd) {Bases de données};

		\end{tikzpicture}
\end{center}

}

\section{Certificats}
\frame{
\frametitle{Certificats}
\begin{block}{Génération de requête}
\$ openssl genrsa -out maclef.pem 2048\\
\$ openssl req -new -key maclef.pem -out marequete.req
\end{block}

\begin{block}{Récupération du certificat}
\begin{enumerate}
\item Faire une demande à l'administrateur de la PKI (julien.legras@etu.univ-rouen.fr)
\item Se rendre sur :\\
\url{http://inf-srv-securechat:8080/ejbca/enrol/server.jsp}
\end{enumerate}
\end{block}
}

\section{Module sécurisé}
\frame{
\frametitle{Module sécurisé}
\begin{block}{Authentification}
Clef RSA 2048 bits certifié par notre PKI\\
$\hookrightarrow$ tunnel SSL entre serveur sécurisé et client sécurisé\\
$\hookrightarrow$ vérification chaîne de certification
\end{block}

\begin{block}{Chiffrement des messages}
AES-256-CBC (Cipher Block Chaining)\\
Clef AES générée et transmises aux destinataires lors de :
\begin{itemize}
\item création de salon
\item ajout/retrait/déconnexion d'utilisateur d'un salon
\end{itemize}
Clef/IV générées à partir d'un mot et d'un sel pseudo-aléatoires (EVP\_BytesToKey + rand)
\end{block}
}

\section{Difficultés rencontrées}
\frame{
\frametitle{Difficultés rencontrées}
\begin{block}{Gestion de projet}
\begin{itemize}
\item format livraison PKI
\item surcharge de travail 
\item difficulté d'intégration des tests de sécurité dans le cahier de recettes
\end{itemize}
\end{block}

\begin{block}{Techniques}
\begin{itemize}
\item apprentissage OpenSSL
\item GTK et threads
\end{itemize}
\end{block}
}

\section{Conclusion}
\frame[allowframebreaks=1.0]{
\frametitle{Conclusion}
\begin{block}{Bilan}
\begin{itemize}
\item Utilisation des documents de projets $\rightarrow$ meilleure planification du développement et actions systématiques

\item Communication, confrontation d'idées dans l'équipe de développement grâce à la méthode agile scrum

\item Apprentissage techniques en sécurité : certification (EJBCA), OpenSSL (bibliothèque et commandes)

\item Interface graphique du C grâce au Vala
\end{itemize}
\end{block}
\framebreak
\begin{exampleblock}{Améliorations envisageables}
\begin{itemize}
\item Gestion de plusieurs serveurs par le client
\item Renouvellement régulier des clefs de chiffrement symétrique
\item Internationalisation de l'application
\item Portage multi-plateformes
\item Amélioration client lignes de commandes (NCurses)
\end{itemize}
\end{exampleblock}


}

\end{document}