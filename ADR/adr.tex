\documentclass[a4paper,11pt,french]{article}
\usepackage[utf8]{inputenc}

\usepackage[T1]{fontenc}
\usepackage[francais]{babel} 
\usepackage[top=2cm, bottom=2cm, left=2cm, right=2cm, includeheadfoot]{geometry} %pour les marges
\usepackage{lmodern}
\usepackage{pict2e}
\usepackage{tikz}	
\usepackage{tikz-uml}
\usepackage{fancyhdr} % Required for custom headers
\usepackage{lastpage} % Required to determine the last page for the footer
\usepackage{extramarks} % Required for headers and footers
\usepackage{graphicx} % Required to insert images
\usepackage{tabularx, longtable}
\usepackage{color, colortbl}
\usepackage{exceltex}

\usepgflibrary{arrows} % for pgf-umlsd

\linespread{1.1} % Line spacing

% Set up the header and footer
\pagestyle{fancy}
\lhead{\textbf{\hmwkClass -- \hmwkSubject \\ \hmwkTitle \\ \hmwkDocName}} % Top left header
\rhead{\includegraphics[width=10em]{logo_univ.png}}
\lfoot{\lastxmark} % Bottom left footer
\cfoot{} % Bottom center footer
\rfoot{Page\ \thepage\ / \pageref{LastPage}} % Bottom right footer
\renewcommand\headrulewidth{0.4pt} % Size of the header rule
\renewcommand\footrulewidth{0.4pt} % Size of the footer rule

\setlength{\headheight}{40pt}

\newcommand{\hmwkTitle}{Chat sécurisé} % Assignment title
\newcommand{\hmwkClass}{Master 1 SSI } % Course/class
\newcommand{\hmwkAuthorName}{Julien Legras} % Your name
\newcommand{\hmwkSubject}{Conduite de projet} % Subject
\newcommand{\hmwkDocName}{Analyse des risques} % Document name

\newcommand{\version}{1.2} % Document version
\newcommand{\docDate}{26 avril 2013} % Document date
\newcommand{\checked}{} % Checker name
\newcommand{\approved}{} % Approver name

\definecolor{gris}{rgb}{0.95, 0.95, 0.95}

\title{
\vspace{2in}
\textmd{\textbf{\hmwkClass :\ \hmwkTitle}}\\
\normalsize\vspace{0.1in}\small{Due\ on\ \hmwkDueDate}\\
\vspace{0.1in}\large{\textit{\hmwkClassInstructor\ \hmwkClassTime}}
\vspace{3in}
}

\author{\hmwkAuthorName}
\date{} % Insert date here if you want it to appear below your name


\begin{document}
\pagestyle{fancy}

\vspace*{5cm}
\begin{center}\textbf{\Huge{\hmwkDocName}}\end{center}
\vspace*{7cm}
	

\fcolorbox{black}{gris}{
\begin{minipage}{15cm}
\begin{tabularx}{10cm}{lXl}
	\bfseries{Version} & & \version\\
	& & \\
	\bfseries{Date} & & \docDate\\
	& & \\
	\bfseries{Rédigé par} & & \hmwkAuthorName \\
	& & \\
	\bfseries{Relu par} & & \checked \\
	& & \\
	\bfseries{Approuvé par} & & \approved \\
	& & \\
\end{tabularx}
\end{minipage}
}

\newpage

%Tableau de mises à jour
\vspace*{1cm}
\begin{center}
\textbf{\huge{MISES À JOUR}}\\
\vspace*{3cm}
	\begin{tabularx}{16cm}{|c|c|X|}
	\hline
	\bfseries{Version} & \bfseries{Date} & \bfseries{Modifications réalisées}\\
	\hline
	0.1 & 23/11/2012 & Création\\
	\hline
	0.2 & 29/11/2012 & Ajout de R1 à R5	\\
	\hline
	0.3 & 29/12/2012 & Ajout de R6 et R7\\
	\hline
	1.1 & 21/02/2013 & Correction du document d'après le retour de Karim ABDELLAH GODARD\\
	\hline
	1.2 & 26/04/2013 & Risque R1 rencontré, ajout du plan d'action effectué\\
	\hline
	\end{tabularx}
\end{center}

%La table des matières
\clearpage
\tableofcontents
\clearpage
\section{Objet}
Document réunissant les différents risques qui pourraient arriver pendant ce projet ainsi que des plans d'actions pour les risques MAJEUR/CRITIQUE.

\section{Terminologie et sigles utilisés}
\begin{itemize}
\item Calcul de la criticité :\\
$$ CRITICITÉ = PROBABILITÉ \times IMPACT $$
\item Table des correspondances PROBABILITÉ/valeur et IMPACT/valeur :

\begin{center}
\begin{tabular}{|l|l|c|l|l|}
\hline
\textbf{Nom probabilité}&\textbf{Valeur}&&\textbf{Nom impact}&\textbf{Valeur}\\
\hline
FAIBLE&2&&MINEUR&4\\
\hline
MAJEUR&3&&MAJEUR&5\\
\hline
FORT&4&&CRITIQUE&6\\
\hline
\end{tabular}
\end{center}
\item []
\item Env: environnement
\item RH : ressources humaines
\end{itemize}
\section{Registre des risques}

\begin{center}
\begin{tabular}{|c|p{3cm}|p{3cm}|l|l|l|c|}
\hline
\textbf{Réf.} & \textbf{Description} & \textbf{Facteurs} & \textbf{Type} & \textbf{Probabilité} & \textbf{Impact} & \textbf{Criticité} \\
\hline
R1&Charge supplémentaire pouvant entraîner un retard&maladie, emploi du temps chargé&Env&MAJEUR&MINEUR&12\\
\hline
R2&Client non-impliqué&emploi du temps chargé&RH&FAIBLE&MINEUR&8\\
\hline
\end{tabular}
\end{center}

\section{Plans d'actions}
\begin{center}
\begin{tabular}{|l|l|l|l|}
\hline
\textbf{Réf.}&\textbf{Action(s) prévue(s)}&\textbf{Action effectuée}&\textbf{Date}\\
\hline
R1&Répartir la charge ou étaler une tâche.&Rallonger la durée du sprint&11/04/2013 (sprint 3)\\
\hline
\end{tabular}
\end{center}
\end{document}