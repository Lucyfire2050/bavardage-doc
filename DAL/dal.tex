\documentclass[a4paper,11pt,french]{article}
\usepackage[utf8]{inputenc}

\usepackage[T1]{fontenc}
\usepackage[francais]{babel} 
\usepackage[top=2cm, bottom=2cm, left=2cm, right=2cm, includeheadfoot]{geometry} %pour les marges
\usepackage{lmodern}
\usepackage{pict2e}
\usepackage{tikz}	
\usepackage{tikz-uml}
\usepackage{fancyhdr} % Required for custom headers
\usepackage{lastpage} % Required to determine the last page for the footer
\usepackage{extramarks} % Required for headers and footers
\usepackage{graphicx} % Required to insert images
\usepackage{tabularx, longtable}
\usepackage{color, colortbl}

\usepgflibrary{arrows} % for pgf-umlsd

\linespread{1.1} % Line spacing

% Set up the header and footer
\pagestyle{fancy}
\lhead{\textbf{\hmwkClass -- \hmwkSubject \\ \hmwkTitle \\ \hmwkDocName}} % Top left header
\rhead{\includegraphics[width=10em]{logo_univ.png}}
\lfoot{\lastxmark} % Bottom left footer
\cfoot{} % Bottom center footer
\rfoot{Page\ \thepage\ / \pageref{LastPage}} % Bottom right footer
\renewcommand\headrulewidth{0.4pt} % Size of the header rule
\renewcommand\footrulewidth{0.4pt} % Size of the footer rule

\setlength{\headheight}{40pt}

\newcommand{\hmwkTitle}{Chat sécurisé} % Assignment title
\newcommand{\hmwkClass}{Master 1 SSI } % Course/class
\newcommand{\hmwkClassInstructor}{Jones} % Teacher/lecturer
\newcommand{\hmwkAuthorName}{Yves Nouafo} % Your name
\newcommand{\hmwkSubject}{Conduite de projet} % Subject
\newcommand{\hmwkDocName}{Architecture du logiciel} % Document name

\newcommand{\version}{0.1} % Document version
\newcommand{\docDate}{23 novembre 2012} % Document date
\newcommand{\checked}{} % Checker name
\newcommand{\approved}{} % Approver name

\newcommand{\fiche}[9] {
	\noindent
\begin{tabular}{|p{3.5cm}| p{1cm} | p{3cm} | p{.5cm} | p{7cm}|} 
\hline
\rowcolor{blue}
\multicolumn{2}{|l|}{\color{white}\bfseries{Nom}} & \multicolumn{3}{l|}{\color{white}\bfseries{#1}}\\
\hline
\multicolumn{2}{|l|}{\bfseries{Acteurs concernés}} & \multicolumn{3}{m{10.5cm}|}{#2}\\
\hline
\multicolumn{2}{|l|}{\bfseries{Description}} & \multicolumn{3}{m{10.5cm}|}{#3}\\
\hline
\multicolumn{2}{|l|}{\bfseries{Préconditions}} & \multicolumn{3}{m{10.5cm}|}{#4}\\
\hline
\multicolumn{2}{|l|}{\bfseries{Evénements déclenchants}} & \multicolumn{3}{m{10.5cm}|}{#5}\\
\hline
\multicolumn{2}{|l|}{\bfseries{Conditions d'arrêt}} & \multicolumn{3}{m{10.5cm}|}{#6}\\
\hline
\rowcolor{gray}
\multicolumn{5}{|c|}{\bfseries{Description du flot d'événements principal}}\\
\hline
\rowcolor{gray}
\multicolumn{3}{|c|}{\bfseries{Acteur(s)}} & \multicolumn{2}{c|}{\bfseries{Système}}\\
\hline
\multicolumn{3}{|p{7.5cm}|}{#7} & \multicolumn{2}{p{7.5cm}|}{#8}\\
\hline
\multicolumn{2}{|l}{\bfseries{Flots d'exceptions}} & \multicolumn{3}{|p{11.5cm}|}{#9}\\
\hline
\end{tabular}
\\
}

\definecolor{gris}{rgb}{0.95, 0.95, 0.95}

\title{
\vspace{2in}
\textmd{\textbf{\hmwkClass :\ \hmwkTitle}}\\
\normalsize\vspace{0.1in}\small{Due\ on\ \hmwkDueDate}\\
\vspace{0.1in}\large{\textit{\hmwkClassInstructor\ \hmwkClassTime}}
\vspace{3in}
}

\author{\hmwkAuthorName}
\date{} % Insert date here if you want it to appear below your name


\begin{document}
\pagestyle{fancy}

\vspace*{5cm}
\begin{center}\textbf{\Huge{\hmwkDocName}}\end{center}
\vspace*{7cm}
	

\fcolorbox{black}{gris}{
\begin{minipage}{15cm}
\begin{tabularx}{10cm}{lXl}
	\bfseries{Version} & & \version\\
	& & \\
	\bfseries{Date} & & \docDate\\
	& & \\
	\bfseries{Rédigé par} & & \hmwkAuthorName \\
	& & \\
	\bfseries{Relu par} & & \checked \\
	& & \\
	\bfseries{Approuvé par} & & \approved \\
	& & \\
\end{tabularx}
\end{minipage}
}

\newpage

%Tableau de mises à jour
\vspace*{1cm}
\begin{center}
\textbf{\huge{MISES À JOUR}}\\
\vspace*{3cm}
	\begin{tabularx}{16cm}{|c|c|X|}
	\hline
	\bfseries{Version} & \bfseries{Date} & \bfseries{Modifications réalisées}\\
	\hline
	0.1 & 23/11/2011 & Création\\
	\hline
	\end{tabularx}
\end{center}

%La table des matières
\clearpage
\tableofcontents
\clearpage
\section{Objet}
\section{Documents applicables et de référecnce}
\section{Terminologie et sigles utilisés}
\section{Configuration requise}
\section{Architecture statique}
\subsection{Structure}
\subsection{Description du constituant "X"}
\subsection{Justifications techniques}

\section{Fonctionnement dynamique}

\section{Traçabilité}

\begin{tikzpicture} 
\begin{umlseqdiag} 
\umlobject[class=A]{a} 
\umlcreatecall[class=B]{a}{b} 
\begin{umlfragment}[type=alt, label=i>5, inner xsep=5] 
\begin{umlcall}[op={tata(i,k)}, dt=7, return=2]{a}{b} 
\end{umlcall} 
\umlfpart[default] 
\begin{umlcall}[op={titi(a,k)}, return=4]{a}{b} 
\end{umlcall} 
\end{umlfragment} 
\end{umlseqdiag} 
\end{tikzpicture}




\end{document}